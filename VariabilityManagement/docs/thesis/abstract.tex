\abstract
In order to reduce time-to-marketing and to improve quality of software products,
several approaches and techniques for economy of scope, mass customization, and
systematic reuse have been proposed. Examples of such approaches include
\emph{Software Product Lines} (SPLs), \emph{Generative Programming Techniques},
and \emph{Software Factories}. Actually, there are commonalities among such
approaches. For instance, the relevance of domain analysis is one common
characteristic, aiming at defining a scope (often in the business sense) in which
reusable assets can be used for generating specific products. Additionally, 
it is a common practice to use \emph{feature modeling} for
representing features that are common to all products within a scope and which
features are optional, being useful for differentiating specific products in a
family. Therefore, aiming to generate specific products, it would be necessary
to: (a) introduce suitable variation points in common assets, (b) develop variant
assets that extend these variation points, and (c) relate features to both common
and variant assets.  In this theis the, we consider \emph{variability
management} as the discipline that guides these activities. In fact, variability management is
an interesting kind of \emph{crosscutting concern}, since certain features
require variation points to be spread in different places of requirements, design, code,
and test artifacts. This crosscutting nature of variability management results in
interesting challenges regarding SPL traceability, evolvability, and product
derivation. As a consequence, several authors have proposed the use of
\emph{aspect-oriented} techniques to better modularize the composition of common
and variant assets of a product line. Besides that, in this thesis we go beyond this composition issue. We mainly consider a 
more encompassing notion of variability management, presenting its semantics as a crosscutting 
concern and describing the contribution of relevant artifacts (such as feature models and
configuration knowledge) in product generation.
\begin{keywords}
Software product lines, Variability management, crosscutting mechanisms
\end{keywords}