\section{Introduction}
The support for variation points enables product customization from a set of
reusable assets~\cite{Pohl:2005aa}. However, variability management, due to its
inherent crosscutting nature, is a common challenge in software product line
(SPL) adoption~\cite{Clements:2001aa,Pohl:2005aa}. First, nontrivial features
often require associated variation points to be scattered through SPL artifacts.
Second, some approaches include product variant and configuration information
inside artifacts.

Both problems can be observed for use case scenario specifications. To minimize
the first problem, several authors have proposed the use of
\emph{aspect-oriented} mechanisms to better modularize the composition of common
and variant scenarios~\cite{moreira-re07,eriksson-splc-2005,pohl-caise-2005},
similar to what has been proposed to source code~\cite{alves-gpce-06,
apel-icse2006}. With respect to the second problem, current
approaches~\cite{favaro-icsr-98,Bertolino:2003aa,Eriksson:2005aa} do not offer a
clear separation between variability management and scenario specification. As a
consequence, in the case where details about product variants are tangled with
use case scenarios, the removal of one variant from the product line requires
changes to all related scenarios. In summary, it is difficult to evolve both
representations.

So in this paper we go beyond the common-variant scenario composition issues and
consider a more encompassing notion of variability management, including
artifacts such as feature models~\cite{gheyi-alloy-06,Czarnecki:2000aa} and
configuration knowledge~\cite{Czarnecki:2000aa,Pohl:2005aa}. We explain this as a
crosscutting phenomenon, using Masuhara and Kiczales work~\cite{Masuhara:2003aa},
and apply this view of \emph{variability management as crosscutting} for
modularizing SPL use case scenarios, providing the necessary separation between
variability and scenario specification concerns. We also formalize the derivation
of product specific scenarios in our approach, as demanded by current SPL
generative practices~\cite{krueger-cacm-200712}. This formalization is based on
our framework for modeling the composition process of scenario variabilities with
feature models, product configuration, and configuration knowledge
(Section~\ref{sec:models}). Besides supporting the mentioned separation of
concerns, this framework helps to precisely specify how to weave the different
representations in order to generate specific scenarios for a SPL member.
Therefore, the main contributions of this work are the following:

\begin{itemize}
\item Characterization of the languages of variability management as a crosscutting concern and, in this way, proposing an approach where variability concerns are separated from other concerns. Although this work focus on requirement artifacts, more specifically use case scenarios, we argue that such separation is also valid for other SPL artifacts. Actually, it has already been claimed for source code~\cite{Alves:2006aa,Apel:2006aa}, without considering the importance of variability artifacts.

\item A framework for modeling the composition process of scenario variability
mechanisms. This framework gives a basis for describing variability mechanisms,
allowing a better understanding of each mechanism. In this work, such framework
is used for modeling the semantics of some scenario variability mechanisms, but
it could be customized for other mechanisms and SPL artifacts.

\item Specification of three scenario variability mechanisms (selection of
optional scenarios, scenario composition, scenario parameterization) using the
modeling framework. Such specification provides a more formal semantic
representation when compared to existing works; this is an important property for
supporting the automatic derivation of product specific artifacts.
\end{itemize}

Since our concept of crosscutting mechanism is based on Masuhara and Kiczales
work~\cite{Masuhara:2003aa}, a smaller contribution of this paper is to apply
their ideas to the languages of variability management,  reinforcing the
generality of their model, which was originally instantiated only for mechanisms
of aspect-oriented programming languages. Based on their view of crosscutting, we
can reason about variability management as a crosscutting concern that involves
different input specifications that contribute to derive a specific member of a
given SPL.

Finally, we evaluate our approach (Section~\ref{sec:evaluation}) by comparing it
to alternative approaches using different product lines. We also relate our work
with other research topics (Section~\ref{sec:related}) and present our concluding
remarks in Section~\ref{sec:conclusions}.