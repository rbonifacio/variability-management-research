\documentclass{acm_proc_article-sp}
%\usepackage{hyperref}
\usepackage{graphicx}
\usepackage{color}
\usepackage{listings}
\usepackage{epsfig}
\usepackage[all]{xy}
\usepackage{multirow}

%include polycode.fmt
%include lhs2tex.sty

%format |->           = "\longmapsto"
%format ^			  = "\otimes"
%format in			  = "\in"
%format id			  = "id"
%format ==            = "=="
%separation 2
%latency 1




\begin{document}
\lstset{language=Haskell, numbers=left,
numberstyle=\tiny,numbersep=5pt,basicstyle=\scriptsize,aboveskip=20pt}

\title{Modeling Scenario Variability with Crosscutting Mechanisms}


\numberofauthors{2}

\author{
\alignauthor
Rodrigo Bonif\'{a}cio\\
       \affaddr{Informatics Center}\\
       \affaddr{Federal University of Pernambuco}\\
       \affaddr{Recife, Brazil}\\
       \email{rba2@@cin.ufpe.br}
\alignauthor
Paulo Borba\\
       \affaddr{Informatics Center}\\
       \affaddr{Federal University of Pernambuco}\\
       \affaddr{Recife, Brazil}\\
       \email{phmb@@cin.ufpe.br}
}

\maketitle

\begin{abstract}
Variability management is a common challenge for Software Product
Line (SPL) adoption, since developers need suitable
mechanisms for specifying and implementing variability
that occurs at different SPL artifacts (requirements, design,
implementation, and test). In this paper, we present a novel approach for
use case scenario variability management, enabling a better
separation of concerns between languages used to manage
variabilities and languages used to specify use case scenarios. The
result is that both representations can be understood and evolved in
a separate way. We achieve such a goal by modeling variability management
as a crosscutting phenomenon, for the reason that artifacts such as feature models,
product configurations, and configuration knowledge crosscut each
other with respect to each specific SPL member. After applying our approach to
different case studies, we achieved a reduction in the size of specifications
and a better feature modularity.
\end{abstract}

\category{D.2.1}{Software Engineering}{Requirements}[Languages,
Methodologies]\
\category{D.2.13}{Software Engineering}{Reusable
Software}

\terms{Design, Documentation}

\keywords{Software product line, variability management, requirements models}

\newdef{definition}{Definition}

%include introduction.tex
%include motivating-example.tex
%include variability-as-crosscutting.tex
%include modeling-framework.tex
%include evaluation.tex
%include related.tex
%include conclusions.tex

%
% ---- Bibliography ----
%

\bibliographystyle{abbrv}
\bibliography{../references/phd-references}
%include apendix.tex

\end{document}
