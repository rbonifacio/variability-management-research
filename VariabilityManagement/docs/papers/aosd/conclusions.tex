%========================
% Conclusions
%========================
\section{Conclusions}\label{sec:conclusions}

In this paper we formally described variability management as a
crosscutting mechanism, considering the contribution
of different input languages that crosscut each other for deriving
specific members of a product line.

We applied this notion of variability management in the context of use
case scenarios, achieving several benefits. First, applying our approach resulted in
a clear separation of concerns between variability and scenario specifications,
allowing both representations to evolve independently. Second, we achieved a reduction
in the size of specifications by composing scenarios in a quantified way. Finally, the
formal specification allowed us to perform several automatic verification in the composition
process. This is an interesting characteristic of our approach, since it can be used
for finding inconsistencies in the final products.

Although our modeling framework was instantiated for representing
scenario variabilities, we believe that it could be applied in
other SPL artifacts. Particularly, optional and parameterized artifacts
are also relevant for non-functional requirements and test cases.
Additionally, our notion of crosscutting was based on a work that states \emph{what}
defines a source code technology as supporting crosscutting modularity. Therefore,
we argue that representing variabilities in source code, using our modeling framework,
is straightforward.

%As future work, we would like to apply our notion of variability management in order
%to identify relationships between
%variabilities at different SPL artifacts (in a first moment, relationships between
%requirements and test artifacts). Such kind of relationships can help in the product
%generation and traceability activities of a SPL.
%The modeling
%framework proposed in this work takes a step in this direction, since the
%composition process used to derive product specific scenarios have been formally represented. 