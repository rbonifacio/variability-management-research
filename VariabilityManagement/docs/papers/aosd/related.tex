%===========================
% Related work
%===========================
\section{Related Work}
\label{sec:related}

Several approaches have been proposed for representing
scenario variability~\cite{Jacobson:1997aa,Eriksson:2005aa,Bertolino:2003aa}. However, in this paper
we focused on PLUC and
PLUSS techniques because they encompass a broad range
of SoC between variability management and scenario specifications.
PLUC presents the lowest level of modularity, since
almost all information related to variability is tangled within
use cases. Although PLUSS partially reduces such a tangling,
by considering the importance of feature modeling, some
dependencies from scenarios to features are still present.

There are works proposed to
represent weaving mechanisms for textual
requirements~\cite{Chitchyan:2007aa,Sillito:2004aa}. However, these approaches offer a narrow support to the \emph{control flow} source of variability, requiring enhancements to integrate their weaving mechanisms to variability management. Besides, generic 
approaches for \emph{aspect-oriented modeling} (AOM) have been applied for 
variability management~\cite{Jayaraman:2007aa,Morin:2008aa,Groher:2008aa}. For instance, Jayaraman and others proposed an approach for modeling variability in UML diagrams~\cite{Jayaraman:2007aa}. Based on their approach, SPL members are generated by means of \emph{graph transformations} specified in MATA (Modeling Aspects using a Transformation Approach). Differently, we proposed a framework for modeling variability as crosscutting mechanisms; and instantiated such a framework for designing variability mechanisms for \emph{textual scenarios}. Moreover, instead of graph transformations, we use an interpreter based approach for SPL member generation--- actually, each reference implementation provided for the weaving process is an interpreter.

%Metrics for quantifying scattering and tangling~\cite{Eaddy:2007aa,Figueiredo:2008aa} have been applied for assessing modularity in aspect-oriented
%programs. 

Finally, we could have used absolute values for quantifying scattering and tangling, as
 in a previous work~\cite{Bonifacio:2008aa}. However, absolute values, such as proposed by Figueiredo and others~\cite{Figueiredo:2008aa}, just reveal if a feature is scattered or not--- without any information about the degree of its scattering. In fact, this limitation hinders the comparison of modularity between different specifications.
As a consequence, we adopted a suite of metrics for quantifying
\emph{degree of scattering} and \emph{degree of focus}~\cite{Eaddy:2007aa}.

%To our knowledge, our work is pioneer in applying crosscutting metrics for evaluating SoC in scenario variability.


