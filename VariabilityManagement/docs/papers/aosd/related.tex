%===========================
% Related work
%===========================
\section{Related Work}
\label{sec:related}

Several approaches have been proposed for representing
scenario variability~\cite{Jacobson:1997aa,Griss:1998aa, Eriksson:2005aa,Bertolino:2003aa}. However, in this paper
we focused on PLUC and
PLUSS techniques because they encompass a broad range
of SoC between variability management and scenario specifications.
PLUC presents the lowest level of modularity, since
almost all information related to variability is tangled within
use cases. Although PLUSS partially reduces such a tangling,
by considering the importance of feature modeling, some
dependencies from scenarios to features are still present.

Regarding to the early representation of aspects, there are works proposed to
represent weaving mechanisms for textual
requirements~\cite{Chitchyan:2007aa,Sillito:2004aa}.  Additionally, generic 
approaches for \emph{aspect-oriented modeling} (AOM) have also been applied for 
variability management~\cite{Morin:2008aa,Groher:2008aa}, despite the fact that he primary goal of these 
works is to propose generic weavers that could be applied to different modeling languages. In this paper, 
we propose a framework for modeling scenario variability management as weavers, 
considering the effect of different languages used in the SPL domain (such as
feature models, configuration knowledge, and product configurations). 
Another difference of our work, comparing it with existing proposals for the
early specification of crosscutting concerns, is that the semantics of our
weavers are expressed using a well defined notion of crosscutting mechanisms.

Finally, metrics for quantifying scattering and tangling~\cite{Eaddy:2007aa,Figueiredo:2008aa} have been applied for assessing modularity in aspect-oriented
systems. We could have used absolute values for quantifying scattering and tangling, as
 in a previous work~\cite{Bonifacio:2008aa}. However, absolute values, such as proposed in~\cite{Figueiredo:2008aa}, just reveals if a feature is scattered or not--- without any information about the degree of its scattering. In fact, this limitation hinders the comparison of modularity between different specifications.
As a consequence, we adopted a suite of metrics for quantifying
\emph{degree of scattering} and \emph{degree of focus}~\cite{Eaddy:2007aa}.
To our knowledge, our work is pioneer in applying crosscutting metrics for evaluating SoC in scenario variability.


