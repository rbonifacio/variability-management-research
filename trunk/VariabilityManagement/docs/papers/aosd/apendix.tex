\appendix
% \section{A configuration knowledge interpreter}

The fragment of code below represents the abstract syntax of the configuration knowledge and the \emph{weavingProcess} function, which corresponds to the implementation of the process shown in Figure~\ref{fig:weave-process}.
 \newpage
 %\begin{small}
 \begin{code}
 type ConfigurationKnowledge = [Configuration]

 data Configuration = Configuration {
  exp :: FeatureExpression,
  tasks :: [Task] 	
 }

 type Task = (SPL -> SPLMember) -> SPLMember

 weavingProcess spl fm pc ck =
    stepRefinement [(t spl) | t <- ts] empty
    where
     ts = concat [tasks c| c <- ck, eval pc (exp c)]
     empty = (emptyInstance spl fc)
     stepRefinement l m = ...   	
 \end{code}
% \end{small}
 
 The \emph{weavingProcess} function applies the appropriate list of tasks ($ts$) 
 for a given product configuration. Therefore, it is necessary to verify  ($eval\ pc\ (exp\ c)$) which expressions ($exp\ c$) are valid for a specific product configuratio ($pc$). The $stepRefinement$ function composes the sequence of tasks that must
 be applied. Supposing $ts = [h,g,f]$, the semantics of
 this hypothetical product would be given by:

 \begin{center}
 $ p\ =\ f\ (spl,\ g\ (spl,\ h\ (spl,\ emptyProduct)))  $
 \end{center}