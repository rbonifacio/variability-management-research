\section{Introduction}
The support for variation points enables product customization from a set of
reusable assets~\cite{Pohl:2005aa}. However, variability management, due to its
inherent crosscutting nature, is a common challenge in software product line
(SPL) adoption~\cite{Clements:2001aa,Pohl:2005aa}. First, nontrivial features
often require associated variation points to be scattered through SPL artifacts.
Second, some approaches include product variant and configuration information
inside artifacts. Both problems can be observed for use case scenario
specifications.

Several authors have proposed the use of \emph{aspect-oriented} mechanisms to
better modularize the specification of crosscutting
concerns~\cite{Moreira:2004aa,Chitchyan:2007aa}. These techniques minimizes the
first problem, since they can be used to modularize the specification of certain
features. However, they do not support the different sources of variability that
occur in SPL requirements. With respect to the second problem, existing
approaches~\cite{Bertolino:2003aa,Eriksson:2005aa} proposed to
scenario variability management do not offer a clear separation between
variability management and scenario specification. As a consequence, in the case
where details about product variants are tangled with use case scenarios, the
removal of one variant from the product line requires changes to all related
scenarios. In summary, it is difficult to evolve both representations.

So in this paper we go beyond the common-variant scenario composition issues and
consider a more encompassing notion of variability management, including
artifacts such as feature models~\cite{Gheyi:2006aa,Czarnecki:2000aa} and
configuration knowledge~\cite{Czarnecki:2000aa,Pohl:2005aa}. We explain this as a
crosscutting phenomenon, using Masuhara and Kiczales work~\cite{Masuhara:2003aa},
and apply this view of \emph{variability management as crosscutting} for
modularizing SPL use case scenarios, providing the necessary separation between
variability and scenario specification concerns. We also formalize the derivation
of product specific scenarios in our approach, as demanded by current SPL
generative practices~\cite{Krueger:2006aa}. This formalization is based on our
framework for modeling the composition process of scenario variabilities with
feature models, product configuration, and configuration knowledge. Besides
supporting the mentioned separation of concerns, this framework helps to
precisely specify how to weave the different representations in order to generate
specific scenarios for a SPL member. Therefore, the main contributions of this
work are the following:

\begin{itemize}
\item Characterization of the languages of variability management as a
crosscutting concern and, in this way, proposing an approach where variability
concerns are separated from other concerns (Section~\ref{sec:svmc}).

%Although this work focus on requirement artifacts, more specifically use case scenarios, we argue that such separation is also valid for other SPL artifacts. Actually, it has already been claimed for source code~\cite{Alves:2006aa,Apel:2006aa}, without considering the importance of variability artifacts.

\item A framework for modeling the composition process of scenario variability
mechanisms (Section~\ref{sec:modeling-framework}). This framework gives a basis
for describing variability as crosscutting mechanisms, {\color{red}at the same
time that the reference implementation provided for each mechanism corresponds to the
essencial part of the environment responsible for mechanizing the composition
processes. However, this paper does not present details about such an environment. Here,
we focus just on its essencial parts.}

\item Specification of three {\color{red}sources of variability for use case
scenarios: variability in function (Section~\ref{sub:pd-weaver}),
variability in data (Section~\ref{sub:bind-weaver}), and variability in control
flow (Section~\ref{sub:sc-weaver})}. This specification provides a more formal
semantic representation when compared to existing works; which is an important property
for supporting the automatic derivation of product specific artifacts.
\end{itemize}

{\color{red}It is important to notice that we do not claim that the sources of
variability (or kinds of composition) for use case scenarios presented here are complete.
However, our modeling framework is able to represent
other interesting sources of variability, such as context-aware adaptability.
Additionally, since the word \emph{scenario} has different meanings in software
engineering, we want to make clear that in the remainder of this paper the term
\emph{scenario} corresponds to textual use case scenarios. We explain the structure
used for specifying scenarios in Section~\ref{sub:spl-uc}}.

Since our concept of crosscutting mechanism is based on Masuhara and Kiczales
work~\cite{Masuhara:2003aa}, a smaller contribution of this paper is to apply
their ideas to the languages of variability management,  reinforcing the
generality of their model, which was originally instantiated only for mechanisms
of aspect-oriented programming languages. Based on their view of crosscutting, we
can reason about variability management as a crosscutting concern that involves
different input specifications that contribute to derive a specific member of a
given SPL. 

Finally, we evaluate our approach (Section~\ref{sec:evaluation}) by comparing it
to alternative approaches using different product lines. We also relate our work
with other research topics (Section~\ref{sec:related}) and present our concluding
remarks (Section~\ref{sec:conclusions}). 