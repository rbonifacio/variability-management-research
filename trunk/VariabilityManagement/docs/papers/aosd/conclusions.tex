%========================
% Conclusions
%========================
\section{Conclusions}\label{sec:conclusions}

In this paper we formally described variability management as crosscutting
mechanisms, considering the contribution of different input languages that
crosscut each other for deriving specific members of a product line. We applied
this notion of variability management in the context of use case scenarios,
achieving several benefits. First, our approach supports a clear separation of
concerns between variability and scenario specifications, allowing both
representations to evolve independently. Second, for different case studies we
improve both feature modularity and scenario cohesion. Finally, we achieved a
reduction in the size of specifications by composing scenarios in a quantified
way. 

% Besides that, investigating the effort for specifying a SPL in existing
% approaches is an issue for future research.

% Although our modeling framework was instantiated for representing
% scenario variabilities, we believe that it could assist the modular design of variability mechanisms to other kinds of artifacts. Particularly, optional and parameterized artifacts
% are also relevant for non-functional requirements and test cases.

% Moreover, our notion of crosscutting was based on a work target at source code definitions of crosscutting. Therefore,
% we argue that representing variabilities in source code, using our modeling framework,
% is straightforward.

\section{Acknowledgments}
We gratefully acknowledge Mehmet Aksit and the anonymous referees
for useful suggestions to improve the paper. This research was partially
sponsored by CNPq (grant CT-INFO 17/2007). 

%As future work, we would like to apply our notion of variability management in order
%to identify relationships between
%variabilities at different SPL artifacts (in a first moment, relationships between
%requirements and test artifacts). Such kind of relationships can help in the product
%generation and traceability activities of a SPL.
%The modeling
%framework proposed in this work takes a step in this direction, since the
%composition process used to derive product specific scenarios have been formally represented. 