\documentclass{acm_proc_article-sp}
%\usepackage{hyperref}
\usepackage{graphicx}
\usepackage{listings}
\usepackage{epsfig}
\usepackage[all]{xy}

%include lhs2tex.fmt
%include lhs2tex.sty

%format |->           = "\longmapsto"
%format ^			  = "\otimes"



\begin{document}
\lstset{language=Haskell, numbers=left,
numberstyle=\tiny,numbersep=5pt,basicstyle=\scriptsize,aboveskip=20pt}

\title{Scenario Variability as Crosscutting}


\numberofauthors{2}

\author{
\alignauthor
Rodrigo Bonif\'{a}cio\\
       \affaddr{Informatics Center}\\
       \affaddr{Federal University of Pernambuco}\\
       \affaddr{Recife, Brazil}\\
       \email{rba2@@cin.ufpe.br}
\alignauthor
Paulo Borba\\
       \affaddr{Informatics Center}\\
       \affaddr{Federal University of Pernambuco}\\
       \affaddr{Recife, Brazil}\\
       \email{phmb@@cin.ufpe.br}
}

\maketitle

\begin{abstract}
Variability management is a common challenge for Software Product
Line (SPL) adoption, since developers need suitable
mechanisms for describing or implementing variability
that might occur at different SPL models (requirements, design,
implementation, and test). In this paper, we present a novel approach for
use case scenario variability management, enabling a better
separation of concerns between languages used to manage
variabilities and languages used to specify use case scenarios. The
result is that both representations can be understood and evolved in
a separated way. We achieve such a goal by modeling variability management
as a crosscutting phenomenon, for the reason that artifacts such as feature models,
product configurations, and configuration knowledge crosscut each
other with respect to a SPL specific member. After applying our approach in
different case studies, we achieved a reduction in the size of specifications,
and a better modularization of feature specifications--- quantified
by feature's degree of scattering and scenario's degree of focus.
\end{abstract}

\category{D.2.1}{Software Engineering}{Requirements}[Languages,
Methodologies]\
\category{D.2.13}{Software Engineering}{Reusable
Software}

\terms{Design, Documentation}

\keywords{Software product line, variability management, requirement models}

%include introduction.tex
%include motivating-example.tex
%include variability-as-crosscutting.tex
%include evaluation.tex
%include related.tex
%include conclusions.tex

%
% ---- Bibliography ----
%

\bibliographystyle{abbrv}
\bibliography{../references/phd-references}

\end{document}
