%===========================
% Related work
%===========================
\section{Related Work}
\label{sec:related}

Our work is linked to the body of research related to SPL
variability management, crosscutting modeling, and use case scenario
composition. This section details some of these approaches, relating
them to the proposed solution.

Pohl et al. argue that variability management should not be
integrated into existing models~\cite{phol-spl-book}. Their proposed
Orthogonal Variability Model (OVM) describes traceability links
between variation points and the conceptual models of a SPL. Our
approach can be applied in conjunction with OVM, since it also
decouple variability representation and offers a crosscutting
mechanism for specific product requirement derivation.

As result of the convergence between model driven development (MDD)
and aspect oriented software development (AOSD), several works were
proposed in order to represent weaving mechanisms using abstract
state machine and activity
diagrams~\cite{noda-aom-2006,thomas-aom-2006}. Our work, on the
other hand, describes weaving mechanisms for scenario specification
using a functional notation. First of all, we believe that the
textual language is the preferred representation for scenario
specification. Second, the use of a functional notation resulted in
a more concise model of the weaving mechanisms.

In order to avoid the problem of \emph{fragile point-cuts}, Rashid
et al. proposed a semantic approach for scenario
composition~\cite{rashid-aosd-2007}. Such approach is based on
natural language processing. Using our scenario composition weaver
(Section~\ref{sub:sc-weaver}), a scenario composition can be
represented using references to \emph{step ids} or \emph{step annotations},
which also reduce the problem of \emph{fragile point-cut}.