
%  $Description: Author guidelines and sample document in LaTeX 2.09$  $Author:
% ienne $ $Date: 1995/09/15 15:20:59 $ $Revision: 1.4 $

\documentclass[times, 11pt,twocolumn]{article}
\usepackage{latex8}
\usepackage{times}



% ------------------------------------------------------------------------- take
% the % away on next line to produce the final camera-ready version
\pagestyle{empty}

% -------------------------------------------------------------------------
\begin{document}

\title{Towards a Crosscutting Approach for Variability Management}

\author{Rodrigo Bonif\'{a}cio \and Paulo Borba\\
Informatics Center \\ Federal University of Pernambuco \\ Recife, Brazil \\
\{rba2, phmb\}@cin.ufpe.br\\ }

\maketitle
\thispagestyle{empty}

\begin{abstract}
   The ABSTRACT is to be in fully-justified italicized text, at the top 
   of the left-hand column, below the author and affiliation 
   information. Use the word ``Abstract'' as the title, in 12-point 
   Times, boldface type, centered relative to the column, initially 
   capitalized. The abstract is to be in 10-point, single-spaced type. 
   The abstract may be up to 3 inches (7.62 cm) long. Leave two blank 
   lines after the Abstract, then begin the main text. 
\end{abstract}



%------------------------------------------------------------------------- 
\Section{Introduction}

In order to reduce time-to-marketing and improve quality of software products,
several approaches and techniques for systematic reuse have been recently
proposed. Examples of such approaches include \emph{Software Product
Lines}~\cite{Pohl:2005aa, Clements:2001aa}, \emph{Generative Programming
Techniques}~\cite{Czarnecki:2000aa}, and \emph{Software
Factories}~\cite{Greenfield:2003aa}. Actually, there is a lot in common among
such approaches. For instance, one common characteristic is the relevance of
domain engineering, which aims at defining a scope (often in the business sense)
in which reusable assets can be used for generating specific products.

Additionally, it is a common practice to use \emph{feature modeling} for
representing features that are common to all products within a scope and which
features are optional, being useful for diferentiating specific products in a
family. Therefore, aiming to generate specific products, it would be necessary
to: (a) introduce suitable variation points in core assets, (b) develop variant assets
that extend these variation points, and (c) relate features to both common and
variant assets.


In this work, we consider \emph{variability management} as the discipline that
guides these activities. Actually, variability management is an interesting kind
of crosscutting concern, since certain features require variation points to be
spread in different places of requirements, design, code, and test artifacts.
This crosscutting nature of variability management results in interesting
challenges regarded to SPL traceability, evolvability, and product derivation. As
a consequence, several authors have proposed the use of \emph{aspect-oriented}
techniques to better modularize the composition of common and variant assets of a
product line~\cite{}. In this thesis we go beyond this composition issue. We
mainly consider a more encompassing notion of variability management, presenting
its semantics as a crosscutting concern and describing the contribution of
relevant artifacts (such as feature models and configuration knowledge) in
product generation.

Our hypotheses is that a clear separation between variability management and
common software engineering artifacts improves SPL evolvability, traceability,
and product derivation. A clear separation means that each SPL model (feature
model, configuration knowledge, SPL use case model, and so on) should focus on
specific concerns. For instance, use case models should represent just the valid
interactions with a software. They should not be enriched to describe variability
space (as proposed in~\cite{Bertolino:2003aa}). The challenge is that, in order
to generate specific products, the clear separation that we are proposing
requires composition processes involving different SPL models. In this thesis,
we define the semantics of composition processes as crosscutting mechanisms.
The elegant notion of crosscutting mechanisms, formalized by Masuhara and
Kiczales~\cite{Masuhara:2003aa}, is used as underlining support for presenting
the semantics of our composition processes. 

Initial results of our approach, applied in the context of representing SPL
variabilities in use case scenarios, revealed to us improvements in both
evolvability and product generation~\cite{Bonifacio:2008aa}. The choice of applying our
approach in this context was motivated because current techniques for scenario variability
management~\cite{Eriksson:2005aa, Bertolino:2003aa} do not present a clear
separation between variability management and scenario specification.
In summary, the contributions of this thesis are threefold

\begin{itemize}
 \item Characterize the broader notation of variability management as
a crosscutting concern and, in this way, propose an approach for
representing it as an independent view of the SPL. Although
this work focuses on requirements artifacts, more specifically
use case scenarios, we argue that such separation is also required
in other SPL models.
 \item A framework for modeling the composition processes of scenario
variability mechanisms. This framework gives a basis for
describing variability mechanisms (such as scenario composition
and parameterization), allowing a better understanding of each mechanism and
highlighting the contribution of each model used in the composition processes.
In this work, such a framework is used for modeling the semantics of scenario variability mechanisms, but it
might be customized for other SPL views.
 \item A deeper evaluation of existing techniques for representing scenario
 variability. Such an evaluation will take into consideration not only the
 support for different variability techniques (parameterization, optional
 scenarios), but also a comparison of existing works with regard to
 SPL evolvability and traceability.  
\end{itemize}

The next section describes our approach, named as \emph{variability management
as crosscutting} (Section~\ref{sec:vmcc}). It also relate some open questions of our
aproach, which we are trying to solve in this thesis. After that,
Section~\ref{sec:evaluation} presents the results of three empirical studies
that we have applied our proposed technique, comparing it with existing works.
In these studies, we have applied \emph{Design Structure Matrices} and a suite
of metrics, addapted from aspect-oriented communit, for quantifying modularity.
We also present in Section~\ref{sec:evaluation} a discussion about how we will
improve our evaluation process. Then, we relate our thesis with existing works
and concludes with \ldots

%-------------------------------------------------------------------------
\section{Variability Management as Crosscutting}\label{sec:vmcc}

%-------------------------------------------------------------------------
\section{Evaluation}\label{sec:evaluation}

%-------------------------------------------------------------------------
\section{Related Work}\label{sec:related}

%------------------------------------------------------------------------- 

\bibliographystyle{latex8}
\bibliography{../references/phd-references}

\end{document}

