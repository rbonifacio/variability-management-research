\chapter{Smart Home Case Study}

This study aimed at comparing feature modularity of PLUSS and MSVCM
specifications. Initially, three different products of the security module of a
Smart Home family~\cite{Pohl:2005aa} were specified. Almost six use cases and
fifteen scenarios are present in each product, with a significant number of
duplicated steps among them. These specifications, available on-line, were used
as input data. Then, six students, organized in groups, were assigned to identify
the commonalities and variabilities among the products, and to restructure the
input specifications. Two SPL specifications were produced, one using the PLUSS
notation and the other one using the MSVCM approach

\section{PLUSS specifications}
In  this Section we present the PLUSS specifications of the Smart Home case
study.  

\begin{figure}[h]
\begin{scriptsize}
  \texttt{
   \begin{tabular}{l}
   	 {\bf Id: SC01} 	\\
     {\bf Description:} Register inhabitant\\
   \end{tabular}
  \begin{center}
  \begin{tabular}{|p{1in}|p{2.5in}|p{2.5in}|}
   \hline
       Step & User Action  & System Response \\ \hline \hline
       1 [RI]	& 
       The home owner selects the register inhabitant
       option in the security configuration menu. & 
       The inhabitant personal form is
       displayed. \\ \hline
       2 [RI] and [Password] & 
       The home owner fills in the inhabitant personal form and selects the
       proceed option. & 
       The system requires the inhabitant password (and
       password confirmation) for getting access to the home. \\ \hline
       2 [RI] and [FingerPrint] & 
       The home owner fills in the inhabitant personal form and selects the proceed option. &
       The system requires the inhabitant inhabitant finger print to capture
       it. \\ \hline
       3 [RI] and [Password] &
       The system requires the inhabitant inhabitant
       finger print to capture it. & 
       The system requests the home owner configuration password. \\ \hline
       3 [RI] and [FingerPrint] & 
       The new inhabitant put your finger in the finger print capture device. &
       The system get the new inhabitant finger print information and asks the
       home owner to proceed. \\ \hline
       4 [RI] and [Password] & 
       The home owner selects the proceed option. & 
       The system requests the home owner configuration password. \\ \hline
       5 [RI] &
      The home owner fills in the configuration password and selects the
      proceed option. & 
      The system register the new inhabitant, allowing he (she) to access the
      smart house. \\ \hline       
  \end{tabular}
  \end{center}
  }
\end{scriptsize}
\caption{Register inhabitant scenario.}
\label{fig:register-inhabitant-pluss}
\end{figure}

\begin{figure}[h]
\begin{scriptsize}
  \texttt{
   \begin{tabular}{l}
   	 {\bf Id: SC02} 	\\
     {\bf Description:} Define the home state\\
   \end{tabular}
  \begin{center}
  \begin{tabular}{|p{1in}|p{2.5in}|p{2.5in}|}
   \hline
       Step & User Action  & System Response \\ \hline \hline
       1 [DHS] + [HS]	& 
       The home owner selects the set home state option in the security configuration menu. & 
       The home state form is displayed, requiring the selection of one of the
       available states HomeState \\ \hline    
       2 [DHS] & The home owner selects one of the configured states. & The
       system requests the home owner configuration password.\\ \hline
       3 [DHS] & The home owner fills in the configuration password and selects
       the proceed option. & The system updates the home state and activates
       the corresponding intrusion and presence policies. \\ \hline
  \end{tabular}
  \end{center}
  }
\end{scriptsize}
\caption{Defines the home state.}
\label{fig:defines-the-home-state-pluss}
\end{figure}

\begin{figure}[h]
\begin{scriptsize}
  \texttt{
   \begin{tabular}{l}
   	 {\bf Id: SC03} 	\\
     {\bf Description:} Define policies for the home state\\
   \end{tabular}
  \begin{center}
  \begin{tabular}{|p{1in}|p{2.5in}|p{2.5in}|}
   \hline
       Step & User Action  & System Response \\ \hline \hline
       1 [DP] + [HS]	& 
       The home owner selects the configure policies for home state option in
       the security configuration menu. & The home state form is displayed,
       requiring the selection of one of the available states HomeState \\
       \hline
       2 [DP] & The home owner selects one of the available states. & The home
       state policy form is displayed, requiring: list of notification numbers,
       audio message, and text message. \\ \hline
       3 [DP] & The home owner fills in the requested information. & The system
       requests the home owner configuration password. \\ \hline 
       4 [DP] & The home owner fills in the configuration password and selects
       the proceed option. & The system updates the home state and activates
       the corresponding intrusion and presence policies. \\ \hline 
  \end{tabular}
  \end{center}
  }
\end{scriptsize}
\caption{Defines the home state.}
\label{fig:defines-the-home-state-pluss}
\end{figure}

\begin{figure}[h]
\begin{scriptsize}
  \texttt{
   \begin{tabular}{l}
   	 {\bf Id: SC04} 	\\
     {\bf Description:} Define rights for the inhabitants\\
   \end{tabular}
  \begin{center}
  \begin{tabular}{|p{1in}|p{2.5in}|p{2.5in}|}
   \hline
       Step & User Action  & System Response \\ \hline \hline
       1 [DefineRights]	& 
       The home owner selects the configure rights for inhabitant option in the
       security configuration menu. & The select an inhabitant form is displayed, 
       requiring the selection of one of the registered inhabitants. \\ \hline
       2 [DefineRights] & The home owner selects one of the registered
       inhabitants & The inhabitant rights form is displayed, requiring the home owner to inform 
       if the inhabitant is able to entering in the house and which environments of the house 
       the inhabitant can change the access control.  \\ \hline
       3 [DefineRights] & The home owner fills in the requested information.  & The system 
       requests the home owner configuration password. \\ \hline 
       4 [DefineRights] & The home owner fills in the configuration password and selects the 
       proceed option. & The system updates the inhabitant rights and activates the corresponding policies. \\ \hline 
  \end{tabular}
  \end{center}
  }
\end{scriptsize}
\caption{Define rights for the inhabitants.}
\label{fig:defines-rights-pluss}
\end{figure}

\begin{figure}[h]
\begin{scriptsize}
  \texttt{
   \begin{tabular}{l}
   	 {\bf Id: SC05} 	\\
     {\bf Description:} Request access to home\\
   \end{tabular}
  \begin{center}
  \begin{tabular}{|p{1in}|p{2.5in}|p{2.5in}|}
   \hline
       Step & User Action  & System Response \\ \hline \hline
       1(a) [RequestAccess] [Password]	& 
       The inhabitant request access to the home at the main entrance. & The
       system requests to the user to pass the smart card. \\ \hline 
       1(b) [RequestAccess] [FingerPrint] & 
       The inhabitant request access to the home at the main entrance. & The
       system requests the user to put your finger in the finger print reader.  \\ \hline 
       (2) [RequestAccess] [Password] & 
       The inhabitant pass the smart card in the card reader. & The system
       requests to the user the password to access the smart home. \\ \hline 
       3(a) [RequestAccess] [Password]&  
       The inhabitant informs the access password.  & 
       The system verifies that the password is valid for the registered
       inhabitant. \\ \hline
       3(b) [RequestAccess] [FingerPrint]&  
       The inhabitant put his (her) finger in the fingerprint reader. & 
       The system recognizes the finger print information as a registered
       inhabitant. \\ \hline
       4 [RequestAccess] & & The system displays the message of success login. \\ \hline
       5 [RequestAccess] & & The system opens the front door and register the occurrence. \\ \hline
       6 [RequestAccess] & & After SecondToClose seconds, the front door is closed. \\ \hline
  \end{tabular}
  \end{center}
  }
\end{scriptsize}
\caption{Request access to home.}
\label{fig:request-access-to-home-pluss}
\end{figure}

\begin{figure}[h]
\begin{scriptsize}
  \texttt{
   \begin{tabular}{l}
   	 {\bf Id: SC06} 	\\
     {\bf Description:} Invalid attempt to home access \\
   \end{tabular}
  \begin{center}
  \begin{tabular}{|p{1.2in}|p{2.3in}|p{2.5in}|}
   \hline
       Step & User Action  & System Response \\ \hline \hline
       1(a) [RequestAccess] [Password]	& 
       The inhabitant request access to the home at the main entrance. & 
       The system requests to the user to pass the smart card. \\ \hline 
       1(b) [RequestAccess] [FingerPrint] & 
       The inhabitant request access to the home at the main entrance. & The
       system requests the user to put your finger in the finger print reader. \\ \hline 
       (2) [RequestAccess] [Password] & 
	   The inhabitant pass the smart card in the card reader. & 
	   The system requests to the user the password to access the smart home.  \\ \hline 
	   3(a) [RequestAccess] [Password]&  
       The inhabitant informs the access password.  & 
		The system verifies that the password is not valid for the registered inhabitant. \\ \hline
       3(b) [RequestAccess] [FingerPrint]&  
       The inhabitant put his (her) finger in the finger print reader. & 
       The system does not recognize the finger print information as a registered inhabitant. \\ \hline 
       (4) [RequestAccess] [Password] & & The system displays the message
       reporting that the inhabitant was not recognized. \\ \hline 
       5(a) [RequestAccess] [Password] & & The system registers the occurrence.\\ \hline 
       5(b) [RequestAccess] [FingerPrint]& & The system register the occurrence, taking a photograph of the not recognized inhabitant. \\ \hline
	   (6) [RequestAccess] [Password] [NumberOfAttempts] & & The system request to
	   the user to pass the smart card again, returning to the previews step. If the number of 
	   attempts is greater than NumberOfAttempts, the system sends an warning message for the home 
	   owner and blocks the inhabitant to enter into the house.  \\ \hline	    	
	   7 [RequestAccess] & & A warning message is sent to the home owner. \\ \hline
  \end{tabular}
  \end{center}
  }
\end{scriptsize}
\caption{Invalid attempt to home access.}
\label{fig:invalid-attempt-to-access-pluss}
\end{figure}

\begin{figure}[h]
\begin{scriptsize}
  \texttt{
   \begin{tabular}{l}
   	 {\bf Id: SC07} 	\\
     {\bf Description:} Guest requests access and the house is empty \\
   \end{tabular}
  \begin{center}
  \begin{tabular}{|p{1.2in}|p{2.3in}|p{2.5in}|}
   \hline
       Step & User Action  & System Response \\ \hline \hline
       1 [GRA]	& 
       A guest request access to the home at main entrance.  & 
       The system requests the guest name and the reason for entering in the house. \\ \hline 
       2 [GRA] & 
       The guest fills in the requested information & 
       The system takes a guest photograph and sends a message to the home
       owner mobile device.  \\ \hline 
       3 [GRA] & &
	   The system asks for the guest to wait a few minutes.  \\ \hline 
       4 [GRA] & 
	   The home owner, using the mobile interface with the security model,
	   authorizes the guest to enter into the house. & 
	   The system displays the message of authorized access. \\ \hline
	   5 [GRA] & &
	   The system opens the front door and register the occurrence. \\ \hline 
	   6 [GRA] & &
	   After SecondToClose seconds, the front door is closed.  \\ \hline 
  \end{tabular}
  \end{center}
  }
\end{scriptsize}
\caption{Guest request access and the house is empty.}
\label{fig:guest-request-access-pluss}
\end{figure}

\begin{figure}[h]
\begin{scriptsize}
  \texttt{
   \begin{tabular}{l}
   	 {\bf Id: SC08} 	\\
     {\bf Description:} Guest requests access but the home owner does not authorize it \\
   \end{tabular}
  \begin{center}
  \begin{tabular}{|p{1.2in}|p{2.3in}|p{2.5in}|}
   \hline
       Step & User Action  & System Response \\ \hline \hline
       1 [GRA]	& 
       A guest request access to the home at main entrance.  & 
       The system requests the guest name and the reason for entering in the house. \\ \hline 
       2 [GRA] & 
       The guest fills in the requested information & 
       The system takes a guest photograph and sends a message to the home
       owner mobile device.  \\ \hline 
       3 [GRA] & &
	   The system asks for the guest to wait a few minutes.  \\ \hline 
       4 [GRA] & 
	   The home owner do not authorize the guest to enter into the house. & 
	   The system displays the message of unauthorized access. \\ \hline
	   5 [GRA] & &
	   The system asks the guest to place a call to the home owner. The scenario finishes.\\ \hline 
  \end{tabular}
  \end{center}
  }
\end{scriptsize}
\caption{Not authorized access of a guest}
\label{fig:guest-request-not-authorized-access-pluss}
\end{figure}

\begin{figure}[h]
\begin{scriptsize}
  \texttt{
   \begin{tabular}{l}
   	 {\bf Id: SC09} 	\\
     {\bf Description:} Configure access to specific environment \\
   \end{tabular}
  \begin{center}
  \begin{tabular}{|p{1.2in}|p{2.3in}|p{2.5in}|}
   \hline
       Step & User Action  & System Response \\ \hline \hline
       1 [CAE]	& 
       The inhabitant selects the configure access rule for specific
       environment option in the security configuration menu. & 
	   The system requests to the user to select a specific environment (room,
	   office, bedroom, kitchen) of the smart home. \\ \hline 
	   2 [CAE] & 
       The inhabitant selects one of the available environments of the smart home.  & 
       The system checks that the inhabitant is able to change the access rule for the selected environment. \\ \hline 
       3 [CAE] & The inhabitant select the access rule to be applied in the
       selected environment. The valid access rules are: restricted and
       unrestricted. & The system requests the inhabitant password.   \\ \hline 
       4 [CAE] & 
	   The inhabitant fills in the password and selects the proceed option. & 
	   The system applies the policy for the selected environment. \\ \hline	    
  \end{tabular}
  \end{center}
  }
\end{scriptsize}
\caption{Configure access to specific environment.}
\label{fig:configure-access-to-environment-pluss}
\end{figure}

\begin{figure}[h]
\begin{scriptsize}
  \texttt{
   \begin{tabular}{l}
   	 {\bf Id: SC10} 	\\
     {\bf Description:} Insufficient privileges to configure access to a specific environment \\
   \end{tabular}
  \begin{center}
  \begin{tabular}{|p{1.2in}|p{2.3in}|p{2.5in}|}
   \hline
       Step & User Action  & System Response \\ \hline \hline
       1 [CAE]	& 
       The inhabitant selects the configure access rule for specific
       environment option in the security configuration menu. & 
	   The system requests to the user to select a specific environment (room,
	   office, bedroom, kitchen) of the smart home. \\ \hline 
	   2 [CAE] & 
       The inhabitant selects one of the available environments of the smart home.  & 
       The system checks that the inhabitant does not have sufficient
       privileges to change the access rule for the selected environment.  \\ \hline 
       3 [CAE] & & The system presents to the user the appropriated message and 
       the scenario finishes. \\ \hline 	    
  \end{tabular}
  \end{center}
  }
\end{scriptsize}
\caption{Insufficient privileges for configure access.}
\label{fig:configure-access-to-environment-pluss}
\end{figure}

\begin{figure}[h]
\begin{scriptsize}
  \texttt{
   \begin{tabular}{l}
   	 {\bf Id: SC11} 	\\
     {\bf Description:} Intrusion detection \\
   \end{tabular}
  \begin{center}
  \begin{tabular}{|p{1.2in}|p{2.3in}|p{2.5in}|}
   \hline
       Step & User Action  & System Response \\ \hline \hline
       1 [ID]	& 
	   The presence sensor identifies that someone is trying to intrude into the
	   smart home. This might be deduced by observing any suspect event (forcing or
       braking) in external doors and windows. & The system emits the configured internal sound alarm.  \\ \hline 
       2 [ID] & &         
       The system places a call to the police department  \\ \hline 
	   (3) [LockBedrooms] & &         
       The system locks the access to the bedrooms.  \\ \hline
	   (4) [TurnOnLights] & &         
       The system turns on external and internal lights. \\ \hline               	    
  \end{tabular}
  \end{center}
  }
\end{scriptsize}
\caption{Intrusion detection.}
\label{fig:intrusion-detection-pluss}
\end{figure}

\begin{figure}[h]
\begin{scriptsize}
  \texttt{
   \begin{tabular}{l}
   	 {\bf Id: SC12} 	\\
     {\bf Description:} Intrusion detection and the house is empty \\
   \end{tabular}
  \begin{center}
  \begin{tabular}{|p{1.2in}|p{2.3in}|p{2.5in}|}
   \hline
       Step & User Action  & System Response \\ \hline \hline
       1 [ID]	& 
	   The presence sensor identifies that someone is trying to intrude into the
	   smart home. This might be deduced by observing any suspect event (forcing or
       braking) in external doors and windows. & The system emits the
       configured internal and external sound alarms. \\ \hline 
       2 [ID] & &         
       The system places a call to the police department and sends a message to the home owner.  \\ \hline 
	   (3) [TurnOnLights] & &         
       The system turns on external and internal lights. \\ \hline               	    
  \end{tabular}
  \end{center}
  }
\end{scriptsize}
\caption{Intrusion detection and the house is empty.}
\label{fig:intrusion-detection-empty-house-pluss}
\end{figure}

\begin{figure}[h]
\begin{scriptsize}
  \texttt{
   \begin{tabular}{l}
   	 {\bf Id: SC13} 	\\
     {\bf Description:} Presence detection \\
   \end{tabular}
  \begin{center}
  \begin{tabular}{|p{1.2in}|p{2.3in}|p{2.5in}|}
   \hline
       Step & User Action  & System Response \\ \hline \hline
       1 [PD]	& 
	   The presence sensor identifies some presence information.  & The system emits the configured external sound alarm.  \\ \hline 
       2 [PD] & &         
       The system places a call to the police department and send a message to the home owner.   \\ \hline 
	   (3) [TurnOnLights] & &         
       The system turns on external and internal lights. \\ \hline               	    
  \end{tabular}
  \end{center}
  }
\end{scriptsize}
\caption{Presence detection.}
\label{fig:presence-detection-pluss}
\end{figure}

\begin{figure}[h]
\begin{scriptsize}
  \texttt{
   \begin{tabular}{l}
   	 {\bf Id: SC14} 	\\
     {\bf Description:} Fire detection \\
   \end{tabular}
  \begin{center}
  \begin{tabular}{|p{1.2in}|p{2.3in}|p{2.5in}|}
   \hline
       Step & User Action  & System Response \\ \hline 
       1 [FD] & The fire sensor identifies a possible fire occurrence (based on
       smoke or temperature elevation). & The system emits the configured
       internal sound alarm. \\ \hline
       2 [FD] & & The system places a call to the fireguard department and
       sends a message to the home owner. \\ \hline
       3 [FD] & & The system unlocks all doors and windows. \\ \hline
       4 [FD] & & The system sends a message to the fire control system, which
       performs its configured reactive actions.  \\ \hline
  \end{tabular}
  \end{center}
  }
\end{scriptsize}
\caption{Fire detection.}
\label{fig:fire-detection-pluss}
\end{figure}


\section{MSVCM specifications}
In this section we present the MSVCM specifications of the Smart Home case
study.

\begin{figure}[h]
\begin{scriptsize}
  \texttt{
   \begin{tabular}{l}
   	 {\bf Scenario SC01} 	\\
     {\bf Description:} Register inhabitant \\
   \end{tabular}
  \begin{center}
  \begin{tabular}{|p{1.2in}|p{2.3in}|p{2.5in}|}
   \hline
       Step & User Action  & System Response \\ \hline
       1 [RI] & The home owner selects the register inhabitant option in the
       security configuration menu. & The inhabitant personal form is
       displayed [RegisterInhabitant]. \\ \hline
       2 [RI] & The home owner fills in the configuration password and selects
       the proceed option. & The system register the new inhabitant, allowing he
       (she) to access the smart house. \\ \hline
  \end{tabular}
  \end{center}
  }
\end{scriptsize}
\caption{Register inhabitant scenario.}
\label{fig:register-inhabitant-msvcm}
\end{figure}

\begin{figure}[h]
\begin{scriptsize}
  \texttt{
   \begin{tabular}{l}
   	 {\bf Advice SC01} 	\\
     {\bf Description:} Register inhabitant with password \\
     {\bf From: RegisterInhabitant} \\
   \end{tabular}
  \begin{center}
  \begin{tabular}{|p{1.2in}|p{2.3in}|p{2.5in}|}
   \hline
       Step & User Action  & System Response \\ \hline
       1 [RI] [Passwordd] & The home owner fills in the inhabitant personal form
       and selects the proceed option. & The system requires the inhabitant password (and password confirmation) 
       for getting access to the home. \\ \hline
       2 [RI] [Password] & The new inhabitant fills in the password (and
       confirmation password) for home access. & The system requests the home owner
       configuration password. \\ \hline
  \end{tabular}
  \end{center}
  }
\end{scriptsize}
\caption{Register inhabitant with password advice.}
\label{fig:register-inhabitant-password-msvcm}
\end{figure}

\begin{figure}[h]
\begin{scriptsize}
  \texttt{
   \begin{tabular}{l}
   	 {\bf Advice SC01} 	\\
     {\bf Description:} Register inhabitant with FingerPrint \\
     {\bf From: RegisterInhabitant} \\
   \end{tabular}
  \begin{center}
  \begin{tabular}{|p{1.2in}|p{2.3in}|p{2.5in}|}
   \hline
       Step & User Action  & System Response \\ \hline
       1 [RI] [FingerPrint] & The home owner fills in the inhabitant personal form and selects 
       the proceed option.  & The system requires the inhabitant inhabitant finger 
       print to capture it. \\ \hline
       2 [RI] [FingerPrint] & The new inhabitant put your finger in the finger print capture 
       device. & The system get the new inhabitant finger print information and
       asks the home owner to proceed.  \\ \hline
       3 [RI] [FingerPrint] & The home owner selects the proceed option. & 
       The system requests the home owner configuration password. \\ \hline
  \end{tabular}
  \end{center}
  }
\end{scriptsize}
\caption{Register inhabitant with fingerprint advice.}
\label{fig:register-inhabitant-fingerprint-msvcm}
\end{figure}
