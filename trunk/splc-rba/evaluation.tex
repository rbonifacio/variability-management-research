\section{Evaluation}
\label{sec:evaluation}

In this section, we present a number of applications of the type system presented in Section~\ref{sec:type-checker}.

\subsection{Paper de AOSD}

\subsection{Checking FM Refactorings}

% refatoramento
A feature model refactoring~\cite{Alves:2006aa} (i) preserves or improves configurability, and (ii) must preserve the well-formedness rules and typing rules.
% foco da secao
So, we should prove two theorems for each refactoring. We already proved that refactorings are sound with respect to the first part~\cite{Gheyi:2006aa-2}. Next we show how to prove them sound with respect to the second part.

% teorema que vamos provar
First, we must prove that a transformation leads a well-formed model into another well-formed model, as formalized by the subsequent theorem. Given the theorem is valid for a transformation, it does not introduce a type error and break a well-formedness rule. \\
\vspace{0.2cm}
				typingRulesPreservation: THEOREM \\
 				$\forall$ m1,m2: FM, $\ldots$ $|$ \\
        syntax($\ldots$) $\wedge$ conditions($\ldots$) $\wedge$ \\
        wellFormed(m1) $\Rightarrow$ wellFormed(m2) \\
\vspace{0.2cm}

% detalhes
The predicate \emph{wellFormed} denotes when a feature model is well-formed and it is explained in Section~\ref{sec:type-checker}. The predicates \emph{syntax} and \emph{conditions} describe the syntax (the similarities and differences of the LHS and RHS models in the refactoring) and conditions of the transformation. In general, we map each construction in the refactoring to each corresponding element in FM semantics. 

% prova de um refatoramento
Next we present a formal argument that Refactoring~\ref{ref:ref2} preserves the typing rules. Suppose that the LHS and the RHS FMs are represented by \emph{m1} and \emph{m2} respectively. We assume that \emph{m1} is well typed. Notice that the new relations in \emph{m2} maintain the property that \emph{m2} is a tree. Moreover, the two new relations (alternative and mandatory) preserve the property that they should have exactly one child. Additionally, the transformation does not introduce new names. Therefore, since \emph{m1} is correct, \emph{m2} does not introduce name conflicts. Finally, all constraints of \emph{m1} are declared in \emph{m2}. Since there is no new feature declared, then the resulting constraints are also well typed. Therefore, Refactoring~\ref{ref:ref2} preserves the typing rules.

% outras leis
The other refactorings can be proved similarly.
