\section{Related Work}\label{sec:related}

\Fix{Leopoldo: dei uma reorganizada na se��o de ferramentas, adicionei o
trabalho do white para comparar e estou iniciando a compara��o com o batory}

% tools
There are several tools that help developers in the process of creating and
maintaning feature models. One of these tools, Feature Modeling Plugin
(FMP)~\cite{Czarnecki:2005aa} cannot detect incorrect
constraints when modeling features, this is only made possible when creating
configurations for a specific product. The techniques described in this paper
provide this support. pure::variants~\cite{Beuche:2003aa} is able to detect some
irregularities on feature models designed with the tool, but none of those
listed on this paper. On the other hand, there is a model validation framework
which can be extended, enabling developers to create additional checks, such
as the ones proposed on this paper. The framework also allows the creation of
quick fixes to solve potential errors found by those checks.

% benavides
Benavides et al.~\cite{Benavides:2005aa} propose an automatic way to analyze FMs,
concerning a number of properties: an enumeration of all configurations, the
number of possible configurations and check whether a FM is consistent. They
present a mapping to transform an extended FM into a Constraint Satisfaction
Problem (CSP) in order to formalize extended FMs using constraint programming.
% relacionamento com os dois trabalhos
Still, these analyses do not consider sound typing, thus our work can be
complementary.

% jules white - splc'08 - melhorar a descri��o e a rela��o dos trabalhos
White et al.~\cite{White-2008aa} present how a constraint solver can derive the
minimal set of feature selection changes to fix an invalid configuration.
Moreover they show how this diagnosis CSP can automatically resolve conflicts
between configuration participant decisions.
% relacionamento com os dois trabalhos
The work is complementary, since these analyses would take place after ours.

% additional approaches in FM refactoring - trujillo, segura (merge)
Trujillo  et al.~\cite{feature-refactoring} present a case study in feature
refactoring, in which the AHEAD Tool Suite is refactored. These refactorings are
not concerned with type soundness. Similarly, Segura et al.~\cite{segura-gttse07}
automate \emph{merging} of FMs lacking typing and well-formedness checking, using
graph transformations as a suitable technology and associated formalism.

% batory - ICSE'09 - refactorings
Thum et al.~\cite{Thum-2009aa} propose an automatic way to reason about edits
to a feature model using SAT solvers. They propose a classification of edits to
feature model. In their definition, our notion of refactoring is split in two
categories. Edits that maintain the same set of products of a FM are called
refactorings. If new products are added and no existing products are removed,
they are called generalizations. They both can be used when the bad smells that we listed are detected.

% Cr�tica do paper do batory - ICSE 2009 - responder
% The main disadvantages of using sound operations is that designers have
% limited possibilities to alter feature models and that all edits that convert
% one model into another one must be known. For example, two feature models
% designed by different domain engineers could be extraordinarily difficult to
% compare

% benavides - dead feature
Concerning bad smells, Trindad et al.~\cite{Trindad:2006aa} already presents a
description of dead (isolated) features, stating a method for detection within a feature model with three implementation alternatives. We go beyond by extending dead features as one situation within a catalog with a number of bad smells, additionally suggesting a way to remove this problem (with type-sound
refactorings).

