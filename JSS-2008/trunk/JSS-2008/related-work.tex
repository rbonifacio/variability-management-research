\section[draft]{Related work}

Modularity issues in aspect-oriented programming have been reported by
different authors~\cite{sullivan-sigsoft-2005, leavens-observers-2002,
steiman-sigplan-2006}. For instance, Clifton and Leavens argue that
existing \emph{aspect-oriented} programming languages do not contribute to
comprehensibility, since they require systems to be studied in their entirety.
In order to improve comprehensibility of AOP systems, they propose a simple set
of restrictions that minimizes this problem~\cite{leavens-observers-2002}. In
their proposed approach, aspects are cathegorized as \emph{observers} ---
aspects that do not change the effective specifications of the modules, or as
\emph{assistents} --- aspects that might change the modules' specifications.
Observers preserve modular reasoning even in the cases that adviced modules do
not make explicit references to them. On the other hand, Leavens and Clifton
argue that, in order to preserve modular reasoning, modules that are adviced by
assistents should make explicitly references for them. A new construction
(\texttt{accept type}) was proposed to indicate that a module (class or
interface) accepts to be adviced by an assistent aspect


Steiman\ldots, \ldots . Other authors foccus on the discussion on how to expose
more stable aspect-oriented interfaces and how to compute module interfaces in
AOP systems. Since these later works are more related to our proposal, in the
remaining of this section we discuss them in more details\ldots


\subsection{Open Modules}

Open Modules is an approach for dealing with modularity issues in 
AOP~\cite{aldrich-ecoop-05}. Besides exposing data structures and 
functions, Open Modules' interfaces can also expose pointcuts denoting 
internal semantic of events. Clients of these modules are able to advice 
only exported pointcuts. Moreover, by exporting a pointcut, the module's 
developer is compromised to maintain the semantics of that pointcuts, 
which, in fact, mitigates the \emph{unantecipated changes} problem.
However, differently of our proposed language, Open Modules does not offers
mechanims for describing which are the responsabilities of aspect developers. As
a consequence, module developers are still able to implement part of a concern
assigned for a different team and they can not assume the existence of any
behavior expected to be modularized as an aspect.
 
\subsection{Crosscut Programming Interfaces}

The Crosscut Programming Interfaces (XPIs) is one attempt for defining design
rules between aspects and advised
code~\cite{sullivan-sigsoft-2005,sullivan-ieee-sw-2006}. Such interfaces specify
which are the exposed join-points of a base code. Although just an abstract XPI
representation was provided in~\cite{sullivan-sigsoft-2005}, Griswold et. al
presented how to implement crosscutting interfaces as syntactic constructs of
AspectJ~\citet{sullivan-ieee-sw-2006}. Based on this approach, design rules are
documented using \emph{abstract pointcut descriptions} and constraints applied to
the base code. These constraints might be written as \emph{declare warning} (or
\emph{declare error}) constructs or just as comments in the source code (when a
constraint can not be expressed as a \emph{declare warning} or \emph{declare
error}). 

XPIs authors argue that the main advantages of their approach is that:
(a) it does not require any new construct in the AspectJ language; and (b) there
is no restriction to the pointcut visibility. In fact, this characteristic
encourage the use of XPIs in sites that are already using the AspectJ language.
However, most of the constraints required for defining the responsibilities of
both class developers and aspect developers can not be checked using the proposed
XPI language (a deeper comparison of our approach against XPIs is presented in
Section~\ref{sec:eval}). Consequently, it is not possible to guarantee, at least
automatically, if certain design rules are being obeyed. 

For instance, we can not
express, using AspectJ constructs, that a call to a method \emph{must} occurs
within the control flow of another method. Actually, we can define such a
pointcut, but if it does not exist,  no error is reported by the AspectJ
compiler. Actually, pointcut constructs in AspectJ were proposed for specifying
points of execution that should be augmented by advices. They were not proposed
for specifying restrictions applied for base code developing.

\subsection{Aspectaware Interfaces}
% Aiming at specifying restrictions in base code using AspectJ like languages,
% developers are able to check if a {\bf non desirable} pointcut occurs in the
% source code (by means of \emph{declare error} or {declare warning} constructs).
% However, there is no construct which can be used for specifying that a specific
% pointcut {\bf must} occur in the source code. However, in order to improve
% modularity between crosscutting and base decisions, it is necessary to define
% what is expected from both teams (not only what aspect developers require, as
% supported by XPIs).

