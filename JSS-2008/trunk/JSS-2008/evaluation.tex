\section{Evaluation}

% Discuss the benefits of applying the DR language in different case studies
% - Health watcher
% - choose one of the cases: TaRGeT, Games, Mobile Media, FLIP, \ldots
% Evaluate the benefits based on metrics
% - Concern metrics or Lattix Metrics

Example 1: Auditing Concern without Design Rules

The auditing concern must be triggered when inserting, removing,
updating, or searching for a complaint.

Two teams working in parallel without design rules: the first one
works on the base code (the ComplaintRepository) and the second one
works on the auditing concern. Such a concern is implemented as an
aspect.

\scriptsize
\begin{lstlisting}[frame=single, caption={Complaint Repository implementation.},label=lst:complaint-repository, language=Java]
public class ComplaintRepository implements IComplaintRepository {

    public int insert(Complaint c) throws RepositoryException, ObjectAlreadyInsertedException {

    }

    public void update(Complaint c) throws RepositoryException, ObjectNotFoundException {

    }

    public void remove(int id) throws RepositoryException, ObjectNotFoundException {

    }

    public Complaint search(int id) throws RepositoryException, ObjectNotFoundException {

    }

}
\end{lstlisting}
\normalsize

Aspect for implementing the auditing concern:

\scriptsize
\begin{lstlisting}[frame=single, caption={Complaint Repository implementation.},label=lst:complaint-repository, language=Java]
public aspect AuditingAspect {

    pointcut auditWhen():
           execution(int ComplaintRepository.insert(Complaint)
        || execution(void ComplaintRepository.update(Complaint)
        || execution(void ComplaintRepository.remove(int)
        || execution(Complaint ComplaintRepository.search(int)

    after() returning(): auditWhen() {
        // audit code.
    }

}
\end{lstlisting}
\normalsize

In the meanwhile, suppose that the teams are working on their
respective concerns (the repository and the auditing concerns). Now,
suppose that the team of the repository changes the following:

\begin{itemize}
    \item the insert method returns void instead of int; and

    \item the remove and search methods should take as parameter the
    Complaint object instead of an id.
\end{itemize}

Such changes might break the aspects...

% =================================== at� aqui entraria na se��o 2...

% ===== se��o 4

Design Rule:

\scriptsize
\begin{lstlisting}[frame=single, caption={Complaint Repository implementation.},label=lst:complaint-repository, language=Java]
dr AuditingDesignRule {

    class ComplaintRepository {

        public int insert(Complaint c) throws RepositoryException, ObjectAlreadyInsertedException {}

        public void update(Complaint c) throws RepositoryException, ObjectNotFoundException {}

        public void remove(int id) throws RepositoryException, ObjectNotFoundException {}

        public Complaint search(int id) throws RepositoryException, ObjectNotFoundException {}

    }

    aspect AuditingAspect {

        pointcut auditWhen():
            execution(int ComplaintRepository.insert(Complaint)
            || execution(void ComplaintRepository.update(Complaint)
            || execution(void ComplaintRepository.remove(int)
            || execution(Complaint ComplaintRepository.search(int)

        % precisa do args aqui...??? Acho que sim...

        after() returning(): auditWhen() {
            xcall(void Auditing.auditComplaint(Complaint));
        }

    }

}
\end{lstlisting}
\normalsize

XPI of this example:

public aspect AuditingXPI {

    pointcut auditWhen():
        execution(int ComplaintRepository.insert(Complaint)
        || execution(void ComplaintRepository.update(Complaint)
        || execution(void ComplaintRepository.remove(int)
        || execution(Complaint ComplaintRepository.search(int)

    pointcut callToAudit(): call(void Auditing.auditComplaint(Complaint));

    declare error: (callToAudit() && !within(AuditingXPI)):
        "Contract violation: the auditComplaint method can not
        be called outside the aspect AuditingXPI";

}

In order to guarantee the contract that classes, aspects and their
respective methods and advices must exist, we can informally
document it like a prose, as mentioned in~\cite{}. Obviously, in
this case the contract is not verified by using the AspectJ
compiler, for instance.

However, we can guarantee the aforementioned contract by using
Aspect-Aware interfaces~\cite{}. Figure TAL illustrates that when a
pointcut does not match any piece of code, the compiler emits a
warning to the developer. Such a warning might be used to alert the
developer that some method must exist, for example.
