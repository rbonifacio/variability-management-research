%===========================
% Related work
%===========================
\section{Related Work}
\label{sec:related}

Several approaches have been proposed for representing
scenario variability~\cite{Jacobson:1997aa,Griss:1998aa, Eriksson:2005aa,Bertolino:2003aa}. However, in this paper
we only compare our crosscutting approach with PLUC and
PLUSS techniques because they encompass a broad range
of SoC between variability management and scenario specifications.
PLUC presents the lowest level of modularity, since
almost all information related to variability is tangled within
use cases. Although PLUSS partially reduces such a tangling,
by considering the importance of feature modeling, some
dependencies from scenarios to features are still present.

Regarding to the early representation of aspects,
there are works proposed to represent weaving mechanisms for textual requirements. Our work also supports the composition of aspects and scenarios specifications. However, differently from the previous works, our mechanisms were proposed for variability management, and considers the effect of different languages used in the SPL domain (such as feature models, configuration knowledge, and product configurations). 
Other works were proposed in order to represent weaving mechanisms using abstract state machines and activity diagrams~\cite{Noda:2006aa,Cottenier:2006aa,Alferez:2008aa}. Our decision for using textual scenario specifications was motivated because this notation is still predominant for documenting requirements--- at least in the Brazilian industry. Additionally, our industrial pattern uses textual scenarios as input for generating test artifacts.

Another difference of our work, comparing it with existing proposals for the early specification of crosscutting concerns, is that the semantics of our weavers are expressed using a well defined notion of crosscutting mechanisms.

In order to avoid the problem of \emph{fragile pointcuts}, Chitchyan
et al. proposed a semantic approach for scenario
composition~\cite{Chitchyan:2007aa}. Such an approach is based on
natural language processing. Using our \emph{evaluateAspect} weaver
(Section~\ref{sub:sc-weaver}), \emph{pointcuts} can be
defined using either references to \emph{step ids} or references to \emph{step annotations};
which, although being more invasive, also reduces the problem of \emph{fragile pointcut}. However, this is an orthogonal problem. We are able to introduce semantic based compositions by implementing the corresponding algorithm in the \emph{match} function, which is used in the reference implementation of  discussed in Section~\ref{sub:sc-weaver}).

Metrics for quantifying scattering and tangling~\cite{Eaddy:2007aa,Figueiredo:2008aa} and
Design Structure Matrices (DSMs)~\cite{Sullivan:2005aa,Lopes:2006aa}
have been applied for assessing modularity in aspect-oriented
systems. In this work, we applied
DSMs in the context of scenario variability management.
Moreover, we customized a suite of metrics for quantifying
the degree of scattering of features and the degree of focus of scenarios.
To our knowledge, our work is unique in applying both DSMs and crosscutting metrics for evaluating SoC in scenario variability.


